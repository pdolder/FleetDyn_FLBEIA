%%%%%%%%%%%%%%%%%%%%%%%%%%%%%%%%%%%%%%%%%%%%%%%%%%%%%%
%% Fleet dynamics models in FLBEIA
%%%%%%%%%%%%%%%%%%%%%%%%%%%%%%%%%%%%%%%%%%%%%%%%%%%%%%

\documentclass[12pt, halfline, a4paper]{ouparticle}

\usepackage[utf8]{inputenc}

%%%%%%%%%%%%%%%%%%%%%%%%%%%%%%%%%%%%
\begin{document}

\title{Working title: Fleet dynamics models in FLBEIA}

\author{
	\name{Paul J. Dolder}
	\address{GMIT}
	\email{paul.dolder@research.gmit.ie}
	\address{CEFAS}
	\email{paul.dolder@cefas.co.uk\thanks{Corresponding author} }
	\and
	\name{Cóilín Minto}
	\address{GMIT}
	\email{coilin.minto@gmit.ie}
	\and
	\name{Dorleta Garcia}
	\address{AZTI}
	\email{dgarcia@azti.es}
}

\abstract{Some text here}

\date{\today}

\keywords{MSE, mixed fisheries, fleet dynamics, RUM, Markov}

\maketitle

\section{Introduction}
\label{intro}

\begin{itemize}

	\item Important for MSEs to take account of mixed fisheries
		interactions
	\item As part of that, need to understand how fleet behaviour might
		effect catch
	\item FLBEIA extension to include commonly used fleet dynamics models	

\end{itemize}

\section{Methods}
\label{meth}

\begin{itemize}
	\item Fit models
		\begin{itemize}
			\item Gravity
			\item Gravity + tradition
			\item Markov
			\item RUM
		\end{itemize}
	\item Condition seasonal FLBEIA model
		\begin{itemize}
			\item 4 seasons
			\item Metiers defined by spatial patterns in catch
			\item Implement fleet dynamics model for Irish otter
				trawlers
		\end{itemize}
\end{itemize}

\section{Results}
\label{res}

\section{Discussion}
\label{dis}

\section{Conclusions}
\label{con}


\begin{notes}[Acknowledgements]
The authors would like to thank...
\end{notes}

\begin{thebibliography}
ggg
\end{thebibliography}


%%%%%%%%%%%%%%%%%%%%%%%%%
\end{document}
