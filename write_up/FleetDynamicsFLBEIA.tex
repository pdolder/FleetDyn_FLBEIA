%%%%%%%%%%%%%%%%%%%%%%%%%%%%%%%%%%%%%%%%%%%%%%%%%%%%%%
%% Fleet dynamics models in FLBEIA
%%%%%%%%%%%%%%%%%%%%%%%%%%%%%%%%%%%%%%%%%%%%%%%%%%%%%%

\documentclass[12pt, halfline, a4paper]{ouparticle}

\usepackage[utf8]{inputenc}
\usepackage{lscape}
\usepackage{rotating}

%%%%%%%%%%%%%%%%%%%%%%%%%%%%%%%%%%%%
\begin{document}

\title{Working title: Fleet dynamics models in FLBEIA}

\author{
	\name{Paul J. Dolder}
	\address{GMIT}
	\email{paul.dolder@research.gmit.ie}
	\address{CEFAS}
	\email{paul.dolder@cefas.co.uk\thanks{Corresponding author} }
	\and
	\name{Cóilín Minto}
	\address{GMIT}
	\email{coilin.minto@gmit.ie}
	\and
	\name{Dorleta Garcia}
	\address{AZTI}
	\email{dgarcia@azti.es}
}

\abstract{Some text here}

\date{\today}

\keywords{MSE, mixed fisheries, fleet dynamics, RUM, Markov}

\maketitle

\section{Introduction}
\label{intro}

Most fisheries worldwide are mixed. Evaluation of management performance is
still based on single-species, failing to take account of fleet dynamics;
technical (mixed-fishery) interactions affect the outcome of management
measures through either discarding unwanted catch or ``choking" of quota when
first limit reached. It's important for MSEs to take account of these
interactions when evaluating management strategies in order to move towards a
more holistic approach to fisheries management.  \\

In order to address this mixed fishery methods have been developed and applied
to numerous case studies (REFs). The mixed-fishery approach models activity of
fleets (vessels of similar physical characteristic) and how the deployment of
fishing effort in different métier (activity defined by similar catch patterns)
contribute to catch of multiple stocks simultaneously. The assumption about the
amount of fishing effort deployed in each of the métier and stability of
catchability for the different stocks determines the outcome for the fishery in
terms of fishing mortality and catch for each stock. The sum of the different
fleets activity (and their catch composition and selectivity) thus provides a
different picture on exploitation of stocks exploited concurrently. \\ 

A limitation in evaluated management strategies from a mixed fishery
perspective is the lack of operating model to account for how fisher behaviour
affects catch of multiple stocks. Location choice is one key decision that
effects catch in mixed fisheries. Different locations have different density of
target and non-target species, therefore choice of where to fish determines how
much of each species caught. This is rarely taken into account in simulations
of management strategies.\\

FLBEIA is a full feedback Management Strategy Evaluation framework that can be
used to take account of mixed-fishery interactions when evaluating different
harvest rules, modelling selectivity improvements and biological and economic
aspects of fisheries. FLBEIA takes a modular approach to modelling
mixed-fisheries, with components for biological and fleet operating models,
management procedures and can take account of full feedback and uncertainty in
management outcome [Insert general diagram of FLBEIA structure]. Generally
applications assume constant fleet dynamics (REFs), with exploitation per unit
of effort for different fleets remaining static inter-annually. This limits
understanding of the impact of fleet dynamics on outcomes for different
fisheries strategies. \\

Here, we extend application of FLBEIA to include commonly used fleet dynamics
models for location choice. This includes the Caddy Gravity model, the
conditional logit Random Utility model and a Markov transition model. We apply
the models to the Celtic Sea demersal fisheries, with location choice for Irish
otter trawlers among nine areas determined by each of the location choice
models and compared to a base case of constant effort. In doing so we show how
different outcomes may be achieved given different assumptions about location
choice. We recommend implementation of different fleet location choice
operating models when undertaking mixed-fishery MSEs in order to incorporate
this important dynamic alongside plausible biological dynamics to better
characterise outcomes for fisheries indicators. \\


\section{Methods}
\label{meth}

\begin{itemize}
	\item Implementation of fleet dynamics models in FLBEIA (pseudo-code)
	\item Conditioning of seasonal FLBEIA model
		\begin{itemize}
			\item 4 seasons
			\item Metiers defined by spatial patterns in catch
			\item Implement fleet dynamics model for Irish otter
				trawlers
		\end{itemize}
	\item Fitting of models 
		\begin{itemize}
			\item Gravity
			\item Gravity + tradition
			\item RUM
			\item Markov 
			\item Model selection for RUM and Markov.
		\end{itemize}
	\item Simulations including closures
		\begin{itemize}
			\item Close a metier, how does effect reallocate
		\end{itemize}
	\item MSE with different fleet dynamics models as OM
\end{itemize}

\section{Results}
\label{res}

Characteristics of the different modelling approaches. \\

How each model is affecting by different species catch rates \\.

Combining inference from the models (different location choice OM, conclusions
about overall strategy. \\


\section{Discussion}
\label{dis}

\section{Conclusions}
\label{con}


\begin{notes}[Acknowledgements]
The authors would like to thank...
\end{notes}

\begin{thebibliography}
ggg
\end{thebibliography}


\newpage

\begin{figure}[!ht]
	\centering
	\includegraphics[width=0.8\linewidth]{figures/Final_Metier_locations}
	\caption{The defined spatial metier} 
	\label{fig:metier}
\end{figure}	

\begin{figure}[!ht]
	\centering
	\includegraphics[width=0.8\linewidth]{figures/Final_Metier_catchcomp}
	\caption{Catch comp for metier} 
	\label{fig:catchcomp}
\end{figure}	

\newpage

\begin{sidewaysfigure}[!ht]
	\centering
	\includegraphics[width=1\linewidth]{figures/Effort_shares}
	\caption{Effort per location model} 
	\label{fig:effort}
\end{sidewaysfigure}	

\newpage

\begin{figure}[!ht]
	\centering
	\includegraphics[width=1\linewidth]{figures/Effort_shares_annual}
	\caption{Annual effort share} 
	\label{fig:effort_an}
\end{figure}	

\begin{figure}[!ht]
	\centering
	\includegraphics[width=1\linewidth]{figures/F_difference}
	\caption{Fishing mortality per scenario} 
	\label{fig:F}
\end{figure}	

\begin{figure}[!ht]
	\centering
	\includegraphics[width=1\linewidth]{figures/IE_Otter_catches}
	\caption{Catches for each stock by Irish Otter trawlers under each
		location model scenario} 
	\label{fig:OtterC}
\end{figure}	

\begin{figure}[!ht]
	\centering
	\includegraphics[width=1\linewidth]{figures/SSB_difference}
	\caption{SSB for each stock under each
		location model scenario} 
	\label{fig:SSB}
\end{figure}	

\begin{figure}[!ht]
	\centering
	\includegraphics[width=1\linewidth]{figures/stock_risks}
	\caption{Stock indicators per scenario} 
	\label{fig:risk}
\end{figure}	

%%%%%%%%%%%%%%%%%%%%%%%%%
\end{document}
